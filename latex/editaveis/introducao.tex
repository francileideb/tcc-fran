\addcontentsline{toc}{chapter}{Introdução}
\chapter*[Introdução]{Introdução}
Em um contexto acadêmico desafiador, onde a pressão dos estudos e responsabilidades é constante, é fundamental reconhecer a importância de atender às demandas emocionais e psicológicas dos estudantes de graduação. Este trabalho propõe uma nova abordagem por meio da implementação de sessões online de Terapia Comunitária Integrativa (TCI) na Universidade de Brasília.

Depressão, ansiedade, transtorno afetivo bipolar, esquizofrenia, entre outros, estão entre os transtornos mentais mais comuns listados pela OMS. Esses distúrbios, caracterizados por pensamentos, percepções, emoções e padrões comportamentais, ressaltam a complexidade e a necessidade de uma abordagem holística à saúde. A saúde mental, comparada à saúde física, surge como um fator obrigatório no bem-estar geral de um indivíduo.\cite{EMMA}Estes desafios impactam não apenas o desempenho acadêmico, mas também a qualidade de vida dos estudantes.

A opção pela TCI fundamenta-se na compreensão de que abordagens terapêuticas tradicionais podem não ser completamente adequadas às necessidades específicas dessa comunidade. Ao contrário, a TCI não apenas busca aliviar os sintomas individuais, mas também coloca um foco substancial na construção de comunidades de apoio e na promoção de um senso de pertencimento, reconhecendo a importância da interconexão social para a saúde mental.

No desenvolvimento da plataforma online, foram empregadas as seguintes tecnologias: Next.js, TypeScript e Tailwind CSS, para assegurar não apenas eficácia, mas também acessibilidade e segurança. O Figma, veio como ferramenta de design e desempenhou um papel importante na criação do projeto da plataforma, permitindo uma visualização do todo ainda na fase inicial.

Além disso, a gestão do projeto foi realizada por meio do Jira e do GitHub, adotando uma abordagem ágil para o acompanhamento hábil de tarefas e marcos, facilitando o desenvolvimento do código e permitindo rastreamento preciso das alterações. A autenticação na plataforma é por meio do OAuth2 do Google, proporcionando um acesso confiável e eficiente. Esta abordagem não apenas assegura a integridade dos dados, mas também possibilita a gestão de calendários através de APIs. A integração com o Google Meet, permite a criação de salas de reunião destinadas à realização das sessões terapêuticas online, ampliando as possibilidades de conexão e engajamento dos participantes.

A metodologia ágil, aliada à análise estatística, assegura uma evolução contínua, adaptando às necessidades identificadas durante a utilização prática da plataforma. Este trabalho não se limita a ser uma solução pontual, mas busca iniciar uma trajetória de pesquisa e desenvolvimento que une tecnologia e psicologia. O compromisso é além da resolução de desafios imediatos, aspirando a construção de um caminho para uma trajetória acadêmica mais saudável, apoiada por ferramentas inovadoras e abordagens terapêuticas centradas na comunidade, com a perspectiva de reduzir as taxas de evasão e promover uma comunidade mais unida.

% \addcontentsline{toc}{chapter}{Introdução}

% Este documento apresenta considerações gerais e preliminares relacionadas 
% à redação de relatórios de Projeto de Graduação da Faculdade UnB Gama 
% (FGA). São abordados os diferentes aspectos sobre a estrutura do trabalho, 
% uso de programas de auxilio a edição, tiragem de cópias, encadernação, etc.

% Este template é uma adaptação do ABNTeX2\footnote{\url{https://github.com/abntex/abntex2/}}.
