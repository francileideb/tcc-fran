\addcontentsline{toc}{chapter}{INTRODUÇÃO}
\chapter*[Introdução]{Introdução}
No contexto desafiador do ambiente acadêmico, sob constante pressão dos estudos e responsabilidades, é crucial reconhecer a importância de atender às necessidades emocionais e psicológicas dos estudantes universitários. Os Transtornos Mentais Comuns (TMC), definidos pela Organização Mundial da Saúde (OMS) e abrangendo sintomas como irritabilidade, insônia e ansiedade, destacam-se como problemas dominantes de saúde mental na população em geral. \cite{RODRIGUES}

Este cenário adquire destaque especialmente entre os jovens universitários, como evidenciado por estudos que apontam um aumento nos índices de Transtornos Mentais Comuns (TMC) neste grupo.\cite{DUFFY} A angústia psicológica muitas vezes está associada a elementos como a falta de alinhamento entre o curso e as expectativas, preocupações sobre o futuro profissional e, em diversos casos, a necessidade de conciliar os horários de estudo com o de trabalho. \cite{BARROS} A transição para o ambiente universitário, caracterizada pelo distanciamento familiar, estabelecimento de novas conexões sociais e aumento da autonomia, surge como um período crítico e desafiador para os estudantes. \cite{PATTON}

O ambiente acadêmico, caracterizado pela necessidade de adaptação e pela inexperiência em relação às dinâmicas acadêmicas, apresenta-se como um contexto propício ao surgimento de desafios relacionados à saúde mental.\cite{RODRIGUES} Pesquisas destacam o ensino superior como um período de intensas mudanças pessoais, sociais, cognitivas e afetivas. \cite{CASTRO} Avaliações sistemáticas da incidência de TMC entre universitários brasileiros abrangem tanto o período anterior à pandemia quanto o cenário pandêmico, evidenciando taxas elevadas que variam de 19\% a 55,3\%.\cite{LOPES}

No Brasil, a discussão sobre a saúde mental dos universitários teve início na década de cinquenta, com a implantação do serviço de higiene mental na Universidade de Pernambuco, destacando uma preocupação contínua com o bem-estar e o desempenho acadêmico dessa população.\cite{CASTRO} A escolha pela Terapia Comunitária Integrativa (TCI) neste trabalho é baseada na compreensão de que abordagens terapêuticas tradicionais podem não ser totalmente adequadas às demandas específicas de uma comunidade acadêmica, principalmente quando não é possível para uma universidade manter um número suficiente de profissionais para atender individualmente à quantidade de alunos presentes nos cursos de graduação.

Além disso, a TCI propõe uma abordagem singular e inovadora, não se limitando a aliviar sintomas individuais, mas concentrando-se substancialmente na construção de comunidades de apoio e promoção do senso de pertencimento.\cite{BARRETO} Isso leva em consideração o contexto no qual a pessoa está inserida, juntamente com os determinantes sociais e de saúde envolvidos na construção da melhoria da saúde mental.

Alguns dos valores que orientam a Terapia Comunitária Integrativa (TCI), como acolhimento, valorização das emoções, horizontalidade das relações e percepção do outro como recurso  \cite{SILVA}, fundamentam a escolha dessa abordagem para fortalecer a comunidade acadêmica e reduzir os índices de evasão nos cursos de graduação. Uma aplicação prática da TCI evidenciou benefícios significativos; um estudo que avaliou os efeitos antes e depois de uma sessão destacou, entre os resultados, a avaliação emocional de uma das participantes. Inicialmente, essa avaliação refletia fragilidade, enquanto após a TCI, ela demonstrou disposição para explorar novas perspectivas. \cite{LEITEePALOS} Essa experiência é um exemplo que ressalta a capacidade transformadora da TCI em situações complexas, proporcionando não apenas um espaço de expressão, mas também potencializando a perspectiva de mudança e crescimento, tanto pessoal quanto comunitário.

Diante de um contexto desafiador, a criação da plataforma online em TypeScript vai além da busca pela eficácia, priorizando também a acessibilidade e o gerenciamento seguro dos links para as sessões de Terapia Comunitária Integrativa (TCI). A escolha estratégica de tecnologias como Next.js, TypeScript e Tailwind CSS, contribui para uma integração eficiente da plataforma ao universo digital mostrando uma experiência terapêutica inovadora e em um ambiente de qualidade.

O uso do Next.js, com suporte nativo ao TypeScript, simplifica o desenvolvimento e proporciona uma estrutura robusta para a criação de páginas dinâmicas e estáticas. Já o TypeScript, com sua tipagem estática opcional, eleva a confiabilidade do código, identificando potenciais erros durante o desenvolvimento e garantindo uma base sólida para a segurança da plataforma.

A implementação da autenticação OAuth 2.0 representa um pilar sólido de segurança, permitindo o gerenciamento seguro dos acessos às sessões de TCI. A escolha estratégica do OAuth 2.0 do Google não só proporciona um método confiável para a autenticação dos usuários, mas também preserva a integridade dos dados, assegurando que apenas usuários autorizados tenham acesso à plataforma. Essa solução consolidada e confiável oferece uma camada adicional de segurança ao ambiente da plataforma.

A sinergia dessas tecnologias não apenas resulta em uma implementação robusta da autenticação, mas também permite a criação de componentes visualmente coesos. O Tailwind CSS, um framework de estilos utilitários, facilita a construção de uma interface de usuário flexível e esteticamente consistente, elevando o aspecto visual da plataforma.

Além das tecnologias fundamentais destacadas, a gestão eficaz do projeto foi assegurada pela adoção de uma abordagem ágil. A coordenação e monitoramento eficientes de tarefas e marcos foram conduzidos por meio de ferramentas essenciais como Jira e GitHub. A implementação de práticas ágeis não apenas permitiu a adaptabilidade às mudanças durante o desenvolvimento, mas também garantiu uma entrega iterativa e consistente, mantendo a transparência e eficiência em todas as etapas do projeto.

Para facilitar as interações e sessões online de Terapia Comunitária Integrativa, foi implementada a integração direta com o Google Meet. Essa funcionalidade permite a criação fácil e rápida de salas de reunião monitoradas pelo terapeuta e dedicadas às sessões terapêuticas, ampliando significativamente as oportunidades de conexão e engajamento entre os participantes. A escolha do Google Meet como solução de videoconferência complementa a abordagem da plataforma, garantindo uma experiência fluida e eficaz para todos os envolvidos.

Assim, a combinação estratégica de práticas ágeis, autenticação segura via OAuth 2.0 do Google e integração com o Google Meet representa uma visão integrada e abrangente no desenvolvimento da plataforma. Cada tecnologia selecionada desempenha um papel fundamental na eficiência, segurança e experiência do usuário, contribuindo para o sucesso geral da iniciativa. Em resumo, a plataforma online em TypeScript transcende a simples resposta ao desafio proposto, sendo um marco que integra inovação, qualidade e segurança.

A metodologia ágil, aliada à análise estatística, visa assegurar uma evolução contínua em resposta às necessidades identificadas durante a utilização prática. Este trabalho propõe-se a ser um ponto de partida, aspirando a inaugurar uma trajetória de investigações futuras sobre a interseção entre tecnologia e psicologia. O objetivo é moldar uma jornada acadêmica mais humanizada, promovendo uma comunidade mais unida e reduzindo as taxas de evasão nos cursos. Esse compromisso não se restringe à resolução de desafios atuais, mas busca construir o caminho para uma trajetória acadêmica mais saudável, apoiada por ferramentas inovadoras e abordagens terapêuticas centradas na comunidade.


% \addcontentsline{toc}{chapter}{Introdução}

% Este documento apresenta considerações gerais e preliminares relacionadas 
% à redação de relatórios de Projeto de Graduação da Faculdade UnB Gama 
% (FGA). São abordados os diferentes aspectos sobre a estrutura do trabalho, 
% uso de programas de auxilio a edição, tiragem de cópias, encadernação, etc.

% Este template é uma adaptação do ABNTeX2\footnote{\url{https://github.com/abntex/abntex2/}}.
