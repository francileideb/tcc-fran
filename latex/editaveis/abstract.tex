\begin{resumo}[Abstract]
 \begin{otherlanguage*}{english}
    The current undergraduate thesis in Electronic Engineering focuses on evaluating the impact of technology on the effectiveness of Integrative Community Therapy (ICT). The main goal is to develop an online platform using technologies like Next.js, TypeScript, and Tailwind CSS to facilitate and optimize therapeutic sessions. The proposed platform aims to act as an interaction space for both users and therapists. Therapists will be able to create and manage online therapeutic groups, while users and undergraduate students at the University of Brasília will have access to find and join these groups. Access to the platform will be enabled through Google's OAuth2 authentication. This approach will allow calendar management via APIs and the creation of meeting rooms on Google Meet, specifically for conducting online therapeutic sessions. The methodology for this project will be based on agile principles, aiming to utilize Platform as a Service (PAAS) for the initial platform implementation. This version will then undergo a detailed statistical analysis, seeking continuous improvements based on the obtained results. Essentially, this work aims to harmonize technology and psychotherapeutic practice, intending to offer a more accessible and effective approach to Integrative Community Therapy. The proposal, therefore, aims to integrate technological tools to enhance and democratize access to community therapeutic approaches.
   \vspace{\onelineskip}
 
   \noindent 
   \textbf{Key-words}: Community Integrative Therapy. CIT. Mental health. Next.js. TypeScript. Tailwind CSS. OAuth2.0.
 \end{otherlanguage*}
\end{resumo}
