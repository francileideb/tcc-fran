\begin{resumo}[Abstract]
 \begin{otherlanguage*}{english}
    The present undergraduate thesis in Electronic Engineering focuses on evaluating the impact of technology on access to Online Community Integrative Therapy (TCI). The main objective is to develop an online platform using technologies such as Next.js, TypeScript, and Tailwind CSS to facilitate and optimize community access to online TCI sessions. The proposed platform's primary function will be managing the availability of community therapists and facilitating participant access. Therapists will be able to create and manage access links for online community groups. University of Brasília users and undergraduate students will have access to the platform to find and participate in groups that fit their schedules, including a video page promoting individual relaxation moments.

    Access to the platform will be enabled through Google's OAuth2 authentication. The choice of this technology will allow calendar management through Application Programming Interfaces (APIs), as well as the creation of meeting rooms on Google Meet for conducting online therapeutic sessions. The methodology adopted in this project will be based on agile principles, seeking to use Platform as a Service (PaaS) for the initial implementation of the online Community Therapy platform. This version will then undergo a detailed statistical analysis of platform usability, aiming to identify and resolve issues, improve and maintain system performance, thus driving continuous innovation.

    At its core, this work aims to harmonize technology and the practice of this advanced psychosocial approach, aspiring to provide intuitive accessibility to online Community Integrative Therapy sessions. The proposal seeks to integrate technological tools to enhance and democratize access to community therapeutic approaches.
   \vspace{\onelineskip}
 
   \noindent 
   \textbf{Key-words}: Community Integrative Therapy. CIT. Mental health. Next.js. TypeScript. Tailwind CSS. OAuth2.0.
 \end{otherlanguage*}
\end{resumo}
