\part{Metodologia}
\chapter{Metodologia do Desenvolvimento}
%\section{Otimizando o Desenvolvimento da Plataforma Online%: Estratégias Eficientes com \textit{Kanban} e Jira}
Desenvolver uma plataforma online escalável e eficiente em \textit{TypeScript} requer uma abordagem integrada e estratégica. A combinação de metodologias ágeis, ferramentas de gerenciamento de projetos e serviços em nuvem desempenha um papel crucial nesse processo.

A metodologia Kanban oferece uma abordagem visual e iterativa que se alinha perfeitamente com o desenvolvimento ágil. Ao criar um quadro \textit{Kanban}, obtém-se uma visualização clara do fluxo de trabalho, desde o \textit{backlog} até as tarefas concluídas, tornando a priorização mais eficaz e concentrando esforços nas atividades essenciais.

O \textit{Jira}, por sua vez, é uma ferramenta versátil de gerenciamento de projetos que complementa o \textit{Kanban} ao criar um quadro em sua plataforma. Atribuição de tarefas, monitoramento de progresso e relatórios detalhados são funcionalidades essenciais, otimizando decisões informadas ao longo do ciclo de desenvolvimento.

No controle de versionamento e automação, o GitHub assume um papel central. Além de vincular \textit{commits} e \textit{pull requests} a tarefas específicas no \textit{Jira}, ele simplifica a automação do fluxo de trabalho através das \textit{GitHub Actions}, promovendo otimização operacional e um desenvolvimento mais fluído. Essas \textit{GitHub Actions} automatizam processos como testes, compilações e implantações contínuas, contribuindo para a otimização e qualidade do desenvolvimento.

A escolha acertada de uma Plataforma como Serviço (PaaS) é crucial para garantir a escalabilidade e o desempenho da plataforma online. Ao optar pelo \textit{Vercel} como a PaaS que funciona de forma nativa com a Next.JS sem a necessidade de muita configuração, é possível oferecer integrações suaves, concentrando-se apenas no desenvolvimento, enquanto a infraestrutura é gerenciada de forma eficaz.

A sinergia entre \textit{Kanban, Jira, GitHub} e \textit{PaaS} é evidente na forma como essas ferramentas se complementam. O \textit{Kanban} no \textit{Jira} guia o desenvolvimento, o \textit{GitHub} gerencia o versionamento e a automação, enquanto a \textit{PaaS} oferece suporte à implantação e escalabilidade da plataforma, criando um ecossistema integrado e otimizado.

A abordagem ágil promovida pelo \textit{Kanban}, em conjunto com as funcionalidades colaborativas do \textit{Jira} e \textit{GitHub}, possibilita integrações contínuas, ajustando o processo com base no \textit{feedback} dos usuários e promovendo melhorias constantes ao longo do ciclo de vida do desenvolvimento da plataforma.

A integração harmoniosa das ferramentas adotadas cria um ambiente coeso que impulsiona a eficiência e a qualidade em todas as fases do projeto de otimização do desenvolvimento de uma plataforma online em \textit{TypeScript}.

\section{%Escolha do
Kanban}
A escolha de implementar o Kanban no desenvolvimento da plataforma online foi fortemente influenciada pela busca por maior visibilidade do projeto, qualidade do software, prazos de entrega mais curtos e uma abordagem sustentável. Caracterizado pela ausência de funções prescritas, entrega contínua e sistema "puxado" de trabalho, o Kanban destaca-se como um método ágil eficiente.

A eficiência nas atividades diárias é vital em todos os processos, e o gerenciamento ágil de projetos incorpora métodos leves para desenvolver e entregar incrementos de produtos com valor em iterações. A plataforma online experimenta um gerenciamento eficaz com melhorias contínuas no fluxo de trabalho. O Kanban baseado em fluxo oferece a oportunidade de aprimorar o que já funciona bem, permitindo uma transição gradual para um processo melhor. Fundamentado em princípios enxutos, o Kanban proporciona benefícios notáveis ao ser integrado ao Jira Software, promovendo a transparência ao exibir rapidamente o status do trabalho e configurar fluxos de trabalho simples ou complexos com eficácia. Além disso, evita gargalos ao estabelecer limites de WIP (Work in Progress) no início dos projetos.

Os quadros Kanban fornecem métricas visuais, como diagramas de fluxo de trabalho cumulativo e gráficos de controle, permitindo o monitoramento do progresso em tempo real e auxiliando na tomada de decisões informadas. A flexibilidade do Kanban em termos de planejamento e execução destaca-se, pois o trabalho é entregue continuamente, em contraste com sprints baseados em tempo. Em resumo, a adoção do Kanban na plataforma online não apenas otimiza o fluxo de trabalho, mas também promove uma gestão mais eficiente e adaptável do projeto.

\section{%Gerenciamento do Projeto com 
Jira}
O Jira Software, ao adotar o método Kanban, é uma ferramenta crucial para aprimorar a eficiência no desenvolvimento da Plataforma online. Este quadro vai além de uma simples lista de tarefas, desempenhando papéis essenciais, como promover a transparência, otimizar fluxos de trabalho e facilitar a detecção de gargalos.

No contexto do desenvolvimento ágil de software, o Kanban é comumente utilizado, e o Jira oferece um painel visual que destaca o estado atual de cada item de trabalho. Ao conduzir um projeto Kanban no Jira, é possível priorizar tarefas, visualizar o fluxo de atividades e minimizar o trabalho em andamento, estabelecendo limites de WIP (Work in Progress) para garantir um fluxo sustentável.

A visualização do fluxo de trabalho é central no Kanban, e o Jira simplifica a criação de um quadro estruturado em estágios, desde o Backlog até o "Done". Reduzir a multitarefa é crucial, e o Jira oferece métricas visuais em tempo real para rastrear o tempo de ciclo e detectar bloqueios no fluxo de trabalho.

Além disso, o Jira integra-se ao GitHub, automatizando o fluxo de trabalho de desenvolvimento. Essa integração elimina atualizações manuais, permitindo que todos os esforços se concentrem no desenvolvimento sem a necessidade de executar manualmente os estágios do quadro Kanban.

Em resumo, os quadros Kanban no Jira Software são fundamentais para maximizar a eficiência, promover a transparência e facilitar o desenvolvimento ágil de software, com a adição de automação através da integração com o GitHub.

%\subsubsection{Configuração do Quadro \textit{Kanban} no Jira}
\section{%Automação de Fluxo de Trabalho com 
GitHub}
%\section{Utilização de Plataformas como Serviço (PaaS)}
\section{%Seleção da 
Plataformas como Serviço (PaaS)}
\section{Configuração e Implantação}
\section{Integração com Bancos de Dados e Outros Serviços}


% \part{Aspectos Gerais}

% \chapter[Aspectos Gerais]{Aspectos Gerais}

% Estas instruções apresentam um conjunto mínimo de exigências necessárias a 
% uniformidade de apresentação do relatório de Trabalho de Conclusão de Curso 
% da FGA. Estilo, concisão e clareza ficam inteiramente sob a 
% responsabilidade do(s) aluno(s) autor(es) do relatório.

% As disciplinas de Trabalho de Conclusão de Curso (TCC) 01 e Trabalho de 
% Conclusão de Curso (TCC) 02 se desenvolvem de acordo com Regulamento 
% próprio aprovado pelo Colegiado da FGA. Os alunos matriculados nessas 
% disciplinas devem estar plenamente cientes de tal Regulamento. 

% \section{Composição e estrutura do trabalho}

% A formatação do trabalho como um todo considera três elementos principais: 
% (1) pré-textuais, (2) textuais e (3) pós-textuais. Cada um destes, pode se 
% subdividir em outros elementos formando a estrutura global do trabalho, 
% conforme abaixo (as entradas itálico são \textit{opcionais}; em itálico e
% negrito são \textbf{\textit{essenciais}}):

% \begin{description}
% 	\item [Pré-textuais] \

% 	\begin{itemize}
% 		\item Capa
% 		\item Folha de rosto
% 		\item \textit{Dedicatória}
% 		\item \textit{Agradecimentos}
% 		\item \textit{Epígrafe}
% 		\item Resumo
% 		\item Abstract
% 		\item Lista de figuras
% 		\item Lista de tabelas
% 		\item Lista de símbolos e
% 		\item Sumário
% 	\end{itemize}

% 	\item [Textuais] \

% 	\begin{itemize}
% 		\item \textbf{\textit{Introdução}}
% 		\item \textbf{\textit{Desenvolvimento}}
% 		\item \textbf{\textit{Conclusões}}
% 	\end{itemize}

% 	\item [Pós-Textuais] \
	
% 	\begin{itemize}
% 		\item Referências bibliográficas
% 		\item \textit{Bibliografia}
% 		\item Anexos
% 		\item Contracapa
% 	\end{itemize}
% \end{description}

% Os aspectos específicos da formatação de cada uma dessas três partes 
% principais do relatório são tratados nos capítulos e seções seguintes.

% No modelo \LaTeX, os arquivos correspondentes a estas estruturas que devem
% ser editados manualmente estão na pasta \textbf{editáveis}. Os arquivos
% da pasta \textbf{fixos} tratam os elementos que não necessitam de 
% edição direta, e devem ser deixados como estão na grande maioria dos casos.

% \section{Considerações sobre formatação básica do relatório}

% A seguir são apresentadas as orientações básicas sobre a formatação do
% documento. O modelo \LaTeX\ \textbf{já configura todas estas opções corretamente},
% de modo que para os usuários deste modelo o texto de toda esta Seção é 
% \textbf{meramente informativo}.

% \subsection{Tipo de papel, fonte e margens}

% Papel -- Na confecção do relatório deverá ser empregado papel branco no 
% formato padrão A4 (21 cm x 29,7cm), com 75 a 90 g/m2.

% Fonte -- Deve-se utilizar as fontes Arial ou Times New Roman no tamanho 12 
% pra corpo do texto, com variações para tamanho 10 permitidas para a 
% wpaginação, legendas e notas de rodapé. Em citações diretas de mais de três 
% linhas utilizar a fonte tamanho 10, sem itálicos, negritos ou aspas. Os 
% tipos itálicos são usados para nomes científicos e expressões estrangeiras, 
% exceto expressões latinas.

% Margens -- As margens delimitando a região na qual todo o texto deverá estar 
% contido serão as seguintes: 

% \begin{itemize}
% 	\item Esquerda: 03 cm;
% 	\item Direita	: 02 cm;
% 	\item Superior: 03 cm;
% 	\item Inferior: 02 cm. 
% \end{itemize}

% \subsection{Numeração de Páginas}

% A contagem sequencial para a numeração de páginas começa a partir da 
% primeira folha do trabalho que é a Folha de Rosto, contudo a numeração em 
% si só deve ser iniciada a partir da primeira folha dos elementos textuais. 
% Assim, as páginas dos elementos pré-textuais contam, mas não são numeradas 
% e os números de página aparecem a partir da primeira folha dos elementos 
% textuais, que se iniciam na Introdução. 

% Os números devem estar em algarismos arábicos (fonte Times ou Arial 10) no 
% canto superior direito da folha, a 02 cm da borda superior, sem traços, 
% pontos ou parênteses. 

% A paginação de Apêndices e Anexos deve ser contínua, dando seguimento ao 
% texto principal.

% \subsection{Espaços e alinhamento}

% Para a monografia de TCC 01 e 02 o espaço entrelinhas do corpo do texto 
% deve ser de 1,5 cm, exceto RESUMO, CITAÇÔES de mais de três linhas, NOTAS 
% de rodapé, LEGENDAS e REFERÊNCIAS que devem possuir espaçamento simples. 
% Ainda, ao se iniciar a primeira linha de cada novo parágrafo se deve 
% tabular a distância de 1,25 cm da margem esquerda.

% Quanto aos títulos das seções primárias da monografia, estes devem começar 
% na parte superior da folha e separados do texto que o sucede, por um espaço 
% de 1,5 cm entrelinhas, assim como os títulos das seções secundárias, 
% terciárias. 

% A formatação de alinhamento deve ser justificado, de modo que o texto fique 
% alinhado uniformemente ao longo das margens esquerda e direita, exceto para 
% CITAÇÕES de mais de três linhas que devem ser alinhadas a 04 cm da margem 
% esquerda e REFERÊNCIAS que são alinhadas somente à margem esquerda do texto 
% diferenciando cada referência.

% \subsection{Quebra de Capítulos e Aproveitamento de Páginas}

% Cada seção ou capítulo deverá começar numa nova pagina (recomenda-se que 
% para texto muito longos o autor divida seu documento em mais de um arquivo 
% eletrônico). 

% Caso a última pagina de um capitulo tenha apenas um número reduzido de 
% linhas (digamos 2 ou 3), verificar a possibilidade de modificar o texto 
% (sem prejuízo do conteúdo e obedecendo as normas aqui colocadas) para 
% evitar a ocorrência de uma página pouco aproveitada.

% Ainda com respeito ao preenchimento das páginas, este deve ser otimizado, 
% evitando-se espaços vazios desnecessários. 

% Caso as dimensões de uma figura ou tabela impeçam que a mesma seja 
% posicionada ao final de uma página, o deslocamento para a página seguinte 
% não deve acarretar um vazio na pagina anterior. Para evitar tal ocorrência, 
% deve-se reposicionar os blocos de texto para o preenchimento de vazios. 

% Tabelas e figuras devem, sempre que possível, utilizar o espaço disponível 
% da página evitando-se a \lq\lq quebra\rq\rq\ da figura ou tabela. 

% \section{Cópias}

% Nas versões do relatório para revisão da Banca Examinadora em TCC1 e TCC2, 
% o aluno deve apresentar na Secretaria da FGA, uma cópia para cada membro da 
% Banca Examinadora.

% Após a aprovação em TCC2, o aluno deverá obrigatoriamente apresentar a 
% versão final de seu trabalho à Secretaria da FGA na seguinte forma:

% \begin{itemize}
% 	\item 01 cópia encadernada para arquivo na FGA;
% 	\item 01 cópia não encadernada (folhas avulsas) para arquivo na FGA;
% 	\item 01 cópia em CD de todos os arquivos empregados no trabalho.
% \end{itemize}

% A cópia em CD deve conter, além do texto, todos os arquivos dos quais se 
% originaram os gráficos (excel, etc.) e figuras (jpg, bmp, gif, etc.) 
% contidos no trabalho. Caso o trabalho tenha gerado códigos fontes e 
% arquivos para aplicações especificas (programas em Fortran, C, Matlab, 
% etc.) estes deverão também ser gravados em CD. 

% O autor deverá certificar a não ocorrência de “vírus” no CD entregue a 
% secretaria. 

