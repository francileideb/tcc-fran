\part{Considerações Finais}
\chapter{Considerações Finais}
\section{Recapitulação dos objetivos}

\section{Contribuições do trabalho e benefícios potenciais do uso da plataforma}
    Este trabalho apresenta contribuições significativas e destaca os benefícios potenciais associados ao uso da plataforma desenvolvida para o gerenciamento eficiente de sessões de Terapia Comunitária Online. Algumas das principais contribuições e benefícios incluem:
    
    1. \textbf{Facilidade de Acesso e Integração:}
       A plataforma oferece uma solução que facilita o acesso às sessões de Terapia Comunitária Online, estreitando os laços dentro da comunidade. Sua capacidade de integração eficaz com outras plataformas proporciona benefícios adicionais, como a criação automática de salas de videoconferência pelo \textit{Google Meet} e adição de eventos na agenda do terapeuta comunitário.
    
    2. \textbf{Ampla Visão dos Horários e Flexibilidade:}
       A plataforma proporciona uma visão abrangente dos horários das rodas online, possibilitando filtrar e ordenar as informações, conferindo maior flexibilidade para os participantes adaptarem as sessões às suas rotinas. Isso promove um acesso mais amplo aos cuidados com a saúde mental, ampliando a participação na Terapia Comunitária Integrativa Online.
    
    3. \textbf{Otimização do Trabalho dos Terapeutas:}
       Ao automatizar a criação de links para as sessões online e simplificar a visualização e gerenciamento dos horários, a plataforma melhora significativamente a produtividade dos terapeutas. Essa otimização, anteriormente realizada manualmente por meio de compartilhamentos em redes sociais, resultava em uma participação frequentemente limitada, se comparada à amplitude que essa abordagem de cuidado com a saúde mental comunitária possibilita.
    
    4. \textbf{Multiplicação do Acesso à Prática Comunitária:}
       A criação da plataforma reforça o compromisso com a prática de cuidados em saúde comunitária, seja presencial ou online. É crucial destacar que a plataforma é disponibilizada de forma totalmente gratuita, multiplicando o acesso a essa poderosa prática de conexão, cura e apoio comunitário.
    
    5. \textbf{Metodologia Ágil e Eficiência no Desenvolvimento:}
       A escolha de empregar uma metodologia ágil alinhada a uma plataforma de gerenciamento de projeto demonstrou ser crucial. Essa abordagem possibilitou um desenvolvimento rápido e eficiente, resultando em uma plataforma capaz de fornecer recursos que amplificam o acesso às sessões de Terapia Comunitária Integrativa Online.
    
    6. \textbf{Diferenciação e Inovação:}
       A automação na criação de links para as sessões de terapia comunitária online e a capacidade de filtrar e ordenar informações de maneira simplificada destacam a inovação da plataforma em oferecer uma solução mais dinâmica e eficaz.
    
    7. \textbf{Utilização da Abordagem para Outros Campos:}
       Os resultados indicam que a abordagem e as tecnologias adotadas no desenvolvimento dessa plataforma podem ser aplicadas em diferentes contextos, permitindo a construção de plataformas personalizadas para diversas finalidades. Isso abre caminho para pesquisas futuras abordarem desafios similares em outras áreas.
    
    Em suma, as contribuições deste trabalho e os benefícios potenciais do uso da plataforma online são expressivos, proporcionando uma solução inovadora e eficiente para o gerenciamento e acesso às sessões de Terapia Comunitária Online, além de indicar possíveis aplicações em diversos cenários.

\section{Limitações e sugestões para trabalhos futuros}
    Embora tenha sido estabelecida uma base sólida para um ambiente online de gerenciamento de sessões de Terapia Comunitária, é fundamental reconhecer algumas limitações e apontar direções para futuras melhorias. Considerando as lições aprendidas com os \textit{usuários testes} deste projeto, identificam-se áreas que representam desafios e oportunidades para futuras melhorias:

      \begin{itemize}
        \item \textbf{Autenticação de Acesso Multifacetada:} a flexibilidade e opções de autenticação para usuários são importantes. Sugere-se explorar maneiras de expandir o suporte para múltiplos provedores \textit{OAuth} 2.0, indo além do Google. Integrar outros provedores permitirá que os usuários escolham métodos de login mais acessíveis e alinhados com suas preferências pessoais.
        
        \item \textbf{Aprimoramento da Experiência do Usuário(UX)}: embora a experiência do usuário seja uma prioridade, sugere-se uma análise mais aprofundada para aprimorar ainda mais a plataforma. Simplificar fluxos e adotar um design centrado no usuário são aspectos fundamentais para criar uma experiência mais envolvente. Detalhes específicos de usabilidade devem ser considerados para manter a plataforma atrativa e intuitiva.
    
      \end{itemize}