\addcontentsline{toc}{chapter}{OBJETIVOS}
\chapter*[Objetivos]{Objetivos}
\section*{Objetivo Geral}
O propósito principal deste trabalho é avaliar o impacto da implementação de uma plataforma online desenvolvida em TypeScript na eficácia da Terapia Comunitária Integrativa (TCI). A pesquisa busca analisar como a tecnologia pode facilitar e aprimorar as sessões de terapia online, considerando fatores como o gerenciamento de salas, acessibilidade, interatividade e efetividade na comunicação entre os participantes, promovendo uma fluidez adequada. Além disso, o projeto visa desenvolver e apresentar uma plataforma funcional que atenda às necessidades específicas da TCI, fornecendo uma base técnica sólida para a realização de sessões terapêuticas de qualidade.

\section*{Objetivos Específicos}
\begin{itemize}
        \item Desenvolver a plataforma online, utilizando tecnologias como Next.js, TypeScript e Tailwind CSS, com foco na facilitação do uso e na otimização das sessões terapêuticas.
      \item Possibilitar aos terapeutas a criação e administração de grupos terapêuticos online, promovendo a interação entre usuários e profissionais.
        \item Permitir que usuários, incluindo alunos de graduação da Universidade de Brasília, encontrem e participem de grupos terapêuticos online de maneira acessível e intuitiva.
       \item Implementar a autenticação OAuth 2.0 do Google para assegurar a segurança e a gestão eficiente de calendários por meio de APIs. Além disso, criar salas de reunião no Google Meet para a realização das sessões terapêuticas online.
       \item Utilizar uma metodologia ágil na implementação inicial da plataforma, empregando plataformas como serviço (PAAS). Submeter a versão inicial a uma análise estatística detalhada para orientar melhorias contínuas.
        \item Realizar uma avaliação do impacto da plataforma por meio de análises estatísticas, buscando compreender seu desempenho e eficácia na facilitação das sessões terapêuticas.
        \item Integrar estrategicamente ferramentas tecnológicas para potencializar e democratizar o acesso a abordagens terapêuticas comunitárias, contribuindo para a inovação na interseção entre engenharia eletrônica e saúde mental.
      \end{itemize}
