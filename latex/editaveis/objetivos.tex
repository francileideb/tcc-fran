\addcontentsline{toc}{chapter}{OBJETIVOS}
\chapter*[Objetivos]{Objetivos}
\section*{Objetivo Geral}
O objetivo deste trabalho é avaliar como uma plataforma online em \textit{TypeScript} pode facilitar o acesso aos grupos de Terapia Comunitária Integrativa Online, destacando a tecnologia no aprimoramento das sessões, promovendo fluidez e adaptabilidade à prática. O projeto visa desenvolver uma plataforma funcional específica para a TCI online, assegurando qualidade nas sessões terapêuticas comunitárias e destacando a importância da tecnologia no cuidado com a saúde mental.

% O propósito principal deste trabalho é avaliar de que maneira a implementação de uma plataforma online, desenvolvida em TypeScript, pode impactar positivamente a eficácia da Terapia Comunitária Integrativa (TCI). A pesquisa busca apresentar como a tecnologia pode ser uma facilitadora e um aprimoramento no gerenciamento das sessões de terapia comunitária online, considerando aspectos cruciais como acessibilidade, interatividade e efetividade na comunicação entre os participantes. Isso promove uma fluidez adequada para essa prática. Além disso, o escopo do projeto inclui o desenvolvimento e apresentação de uma plataforma funcional projetada para atender às necessidades específicas da TCI, garantindo uma base técnica robusta para a realização de sessões terapêuticas de alta qualidade.

% Essa abordagem não apenas destaca a relevância da tecnologia na facilitação do processo terapêutico online, mas também reforça o compromisso em adaptar a plataforma às particularidades e exigências necessárias para a promoção do cuidado com a saúde mental.

\section*{Objetivos Específicos}
\begin{itemize}
        \item Desenvolver a plataforma online, utilizando tecnologias como \textit{Next.js}, \textit{TypeScript} e \textit{Tailwind CSS}, com foco na facilitação e otimização do ingresso de usuários as sessões de TCI online.
        \item Possibilitar aos terapeutas a criação e administração de sessões online e permitir que usuários, especialmente os alunos de graduação da Universidade de Brasília, encontrem sessões que adequam a suas rotinas.
        \item Integrar ferramentas tecnológicas para potencializar e democratizar o acesso a abordagens terapêuticas comunitárias..
       \item Utilizar uma metodologia ágil na implementação inicial da plataforma, empregando plataformas como serviço (PaaS).
       \item Implementar a autenticação \textit{OAuth} 2.0 do Google para assegurar a segurança e a gestão eficiente por meio de APIs.
%       \item Submeter a versão inicial a uma análise estatística detalhada para orientar melhorias contínuas.
        \item Realizar uma avaliação do impacto da plataforma por meio de análises estatísticas de usabilidade, buscando compreender seu desempenho e eficácia na facilitação das sessões terapêuticas comunitárias.
\end{itemize}
