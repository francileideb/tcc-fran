\begin{apendicesenv}

\partapendices
     \chapter{Termos de Uso}
     \label{appendix:termos}
        1. Aceitação dos Termos de Uso da Plataforma 
        
        Ao acessar ou utilizar a Plataforma de Terapia Comunitária Integrativa Online, você concorda em cumprir estes Termos de Serviço e todas as leis e regulamentos aplicáveis. Os Termos e Condições a seguir regem seu acesso e uso de nossa plataforma online, que é acessível gratuitamente e está aberta a todos, respeitando todas as etnias, gêneros, orientações sexuais, deficiências, religiões ou nacionalidades. Para uma melhor usabilidade, recomendamos a navegação na versão web. Se você não concordar com algum destes termos, você não tem permissão para usar a Plataforma.
        
        2. Conta de Usuário, Conduta e Compromissos no Uso da Plataforma
        
        Você concorda em fornecer informações precisas ao criar uma conta e atualizar essas informações regularmente. Você é responsável pela segurança de sua conta.
        Conduta do Usuário: Você concorda em usar a Plataforma de forma ética e em conformidade com estes Termos de Serviço e todas as leis aplicáveis.
        Você concorda, confirma e reconhece que é totalmente responsável por todas as atividades realizadas usando sua conta. Compromete-se a não usar a conta ou o Acesso à Conta de qualquer outra pessoa.
        Você concorda e se compromete a não interferir nos sistemas, serviços, servidores, redes e infraestrutura da Plataforma, protegendo a integridade do ambiente terapêutico online.
        Você autoriza ser redirecionado para conteúdos oferecidos por outras plataformas, como \textit{YouTube} e \textit{Google Meet}, na intenção de aprimorar a sua experiência nos serviços prestados pela Plataforma de Terapia Comunitária Integrativa.
        
        3. Perfis de Usuários
        
        Terapeuta:
        Como Terapeuta, ao utilizar o perfil na Plataforma, você concorda em obrigatoriamente possuir formação em Terapia Comunitária Integrativa (TCI), credenciada e capacitada, após concluir com êxito a educação, treinamentos necessários e requisitos práticos, conforme aplicável pelos Polos Formadores e de Cuidado vinculados e credenciados pela Associação Brasileira de Terapia Comunitária Integrativa (ABRATECOM) ou pela Associação Europeia de TCI (AETCI).
        Como Terapeuta, você terá permissões de edição na plataforma, incluindo a capacidade de criar "Novos Grupos". Ao criar "Novos Grupos", você, Terapeuta, concorda em nomear os grupos de forma ética e garantir que o conteúdo seja apropriado para a natureza terapêutica proposta pela Plataforma.
        Você, Terapeuta, não terá acesso aos dados do perfil "Participante" e concorda em respeitar a privacidade dos usuários.
        Você, Terapeuta, concorda que será o organizador da sala de videoconferência do \textit{Google Meet} e possui a responsabilidade de autorizar a entrada dos Participantes.
        Você concorda que como terapeuta presta um serviço independente e de forma gratuita para promover a criação de redes comunitárias. Sendo o papel da Plataforma limitado a permitir o gerenciamento das rodas de terapia mediadas pelo Terapeuta, sendo os próprios, responsáveis pela execução dessas rodas.
        
        Participante:
        
        Como participante, você terá acesso apenas à visualização na Plataforma.
        Você pode navegar pelas páginas, visualizar grupos criados por Terapeutas e usufruir das informações e interatividades disponíveis.
        Como participante, ao acessar as rodas online pelo \textit{Google Meet}, você concorda em seguir as Regras da Terapia Comunitária Integrativa, promovendo um ambiente de escuta ativa e respeito mútuo. Essas regras incluem fazer silêncio durante a fala de outros participantes, falar apenas sobre si mesmo e suas experiências, evitar aconselhamentos e discursos, e contribuir de maneira culturalmente enriquecedora.
        Você, Participante, concorda que deverá solicitar acesso e aguardar no \textit{lobby} de espera até que o Terapeuta autorize sua entrada.
        
        4. Modificações nos Termos de Uso
        
        A Plataforma reserva-se o direito de modificar estes Termos de Uso a qualquer momento. As alterações entrarão em vigor imediatamente após a publicação na Plataforma.
        
        5. Rescisão de Conta
        
        A Plataforma pode rescindir ou suspender sua conta e acesso à Plataforma por qualquer motivo, sem aviso prévio.
        
        6. Disposições Gerais
        
        Estes Termos de Uso constituem o acordo integral entre você e a Plataforma de Terapia Comunitária Integrativa e regem o uso da Plataforma. Ao utilizar nossa Plataforma, você está comprometendo-se a participar de uma comunidade terapêutica respeitosa e ética. Esperamos que a experiência na Plataforma contribua positivamente para o seu bem-estar e desenvolvimento pessoal.

    \chapter{{P}olítica de Privacidade}
    \label{appendix:privacidade}
        Este documento estabelece os termos e condições referentes à coleta e utilização de dados pessoais dos usuários da Plataforma de Terapia Comunitária Integrativa Online. Ao utilizar a Plataforma, o usuário concorda com as disposições descritas neste Termo de Privacidade.
    
        Coleta de Informações Pessoais:
        Conscientemente, recebemos e armazenamos as informações fornecidas pelos usuários ao conectar-se pelo \textit{OAuth} 2.0 do \textit{Google}. Atualmente, essas informações estão associadas ao endereço de e-mail do Google, necessário para o login e acesso ao conteúdo da Plataforma de Terapia Comunitária Integrativa.
        
        Coleta de Informações Não Pessoais:
        Quando o usuário visita a Plataforma, nossos servidores registram automaticamente informações enviadas pelo navegador. Esses dados incluem o endereço IP do dispositivo, tipo e versão do navegador, tipo e versão do sistema operacional, páginas visitadas, tempo gasto nessas páginas, horários e datas de acesso, e outras estatísticas.
        
        Gerenciamento de Informações Pessoais:
        Os usuários podem gerenciar suas informações pessoais na seção "Gerenciar conta" da Plataforma de Terapia Comunitária Integrativa. As opções incluem sair da conta em dispositivos ativos e excluir a conta. Algumas informações podem ser guardadas por um período determinado após a exclusão, sendo posteriormente excluídas de forma permanente.
        
        Uso e Processamento das Informações Coletadas:
        As informações coletadas são utilizadas para personalizar a experiência do usuário na Plataforma de Terapia Comunitária Integrativa, aprimorar a usabilidade, melhorar o conteúdo disponível e facilitar a criação de links para salas de videoconferência e eventos no calendário associado à conta do Google. As informações não pessoais são usadas para identificar possíveis \textit{bugs} e gerar estatísticas de uso, sem identificar usuários específicos.
        
        Uso de \textit{Cookies}:
        A Plataforma de Terapia Comunitária Integrativa utiliza \textit{cookies} para personalizar a experiência online, coletar informações estatísticas e operar de maneira eficiente. \textit{Cookies} são atribuídos exclusivamente ao usuário e podem ser lidos apenas por servidores no domínio emissário do \textit{cookie}.
        
        Segurança da Informação:
        Implementamos camadas de segurança para proteger as informações pessoais contra acesso não autorizado. Reconhecemos limitações de segurança na Internet e, embora nos esforcemos para proteger as informações, não podemos garantir a segurança, integridade e privacidade durante a transmissão pela Internet.
        
        Violação de Dados:
        Em caso de comprometimento da segurança da plataforma ou divulgação não autorizada de informações pessoais, devido a atividades externas, nos reservamos o direito de tomar medidas apropriadas. Faremos esforços razoáveis para notificar os usuários afetados, caso exista risco de danos ou se a notificação for exigida por lei.
        
        Mudanças e Alterações:
        Reservamo-nos o direito de modificar esta política de privacidade a qualquer momento, publicando uma versão atualizada na Plataforma de Terapia Comunitária Integrativa. O uso contínuo após tais alterações constituirá consentimento para as mesmas.
        
        Lei Geral de Proteção de Dados (LGPD):
        A Plataforma foi desenvolvida considerando os principais princípios da LGPD, legislação brasileira que regulamenta o tratamento de dados pessoais por organizações públicas e privadas. Tais princípios incluem a transparência no tratamento dos dados, respeito à privacidade, finalidade específica do uso dos dados, necessidade e segurança dos dados coletados. Essa medida visa garantir a conformidade com as normas de proteção de dados.
        
        Aceitação desta Política:
        Ao utilizar a Plataforma de Terapia Comunitária Integrativa Online, você reconhece ter lido e concorda com os termos desta Política. Caso não concorde, não está autorizado a usar ou acessar a Plataforma.
    
    \chapter{Estrutura do Projeto}
    \label{appendix:estrutura}
        \begin{verbatim}
        |____pnpm-lock.yaml
        |____middleware.ts
        |____.env.local
        |____.gitignore
        |____.git
        |____package.json
        |____.env
        |____components.json
        |____tsconfig.json
        |____postcss.config.js
        |____.eslintrc.json
        |____next-env.d.ts
        |____README.md
        |____tailwind.config.ts
        |____next.config.js
        |____app
        | |____sign-up
        | | |____[[...sign-up]]
        | | | |____page.tsx
        | |____terms-of-use
        | | |____page.tsx
        | |____favicon.ico
        | |_____posthog
        | | |____posthog.ts
        | | |____PHProvider.tsx
        | |____terapeuta
        | | |____navbar.tsx
        | | |____data-table.tsx
        | | |____columns.tsx
        | | |____docs
        | | | |____page.tsx
        | | |____newGroup
        | | | |____page.tsx
        | | | |____form.tsx
        | | |____layout.tsx
        | | |____page.tsx
        | | |____dropdown-menu.tsx
        | |____privacy-policy
        | | |____page.tsx
        | |_____trpc
        | | |____client.ts
        | | |____serverClient.ts
        | | |____Provider.tsx
        | |_____lib
        | | |____utils.ts
        | |____participante
        | | |____navbar.tsx
        | | |____data-table.tsx
        | | |____columns.tsx
        | | |____docs
        | | | |____page.tsx
        | | |____layout.tsx
        | | |____dialog.tsx
        | | |____page.tsx
        | | |____dropdown-menu.tsx
        | |_____components
        | | |____ui
        | | | |____popover.tsx
        | | | |____toaster.tsx
        | | | |____label.tsx
        | | | |____navigation-menu.tsx
        | | | |____use-toast.ts
        | | | |____calendar.tsx
        | | | |____avatar.tsx
        | | | |____dialog.tsx
        | | | |____table.tsx
        | | | |____button.tsx
        | | | |____toast.tsx
        | | | |____checkbox.tsx
        | | | |____dropdown-menu.tsx
        | | | |____input.tsx
        | | | |____form.tsx
        | | |____Player.tsx
        | |____layout.tsx
        | |____api
        | | |____trpc
        | | | |____[trpc]
        | | | | |____route.ts
        | | |____clerk
        | | | |____route.ts
        | |____sign-in
        | | |____[[...sign-in]]
        | | | |____page.tsx
        | |____page.tsx
        | |____globals.css
        |____prisma
        | |____schema.prisma
        |____server
        | |____routers
        | | |____videos.ts
        | | |____groups.ts
        | | |____users.ts
        | |____trpc.ts
        | |____index.ts
        | |____db.ts
        |____public
        | |____vercel.svg
        | |____next.svg
        \end{verbatim}       

\end{apendicesenv}
