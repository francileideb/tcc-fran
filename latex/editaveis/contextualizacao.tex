\part{Contextualização}

\chapter{Terapia Comunitária Integrativa}
A Terapia Comunitária Integrativa (TCI) surgiu em 1985 no bairro Pirambu, Fortaleza, CE, a partir de uma rede comunitária informal. Esse movimento nasceu da união de moradores, conectados por experiências compartilhadas, resultando na criação de uma prática terapêutica significativa. O fundador, Adalberto de Paula Barreto, doutor em psiquiatria, teologia e antropologia, desenvolveu a TCI com uma base interdisciplinar sólida, ancorada nos pilares do pensamento sistêmico, pragmática da comunicação humana, antropologia cultural, pedagogia de Paulo Freire e resiliência.\cite{BARRETO}

A metodologia da TCI é estruturada em fases como acolhimento, seleção da inquietação, contextualização, partilha de experiências e conclusão positiva. Criando um espaço protegido para a expressão e ressignificação das vivências individuais e coletivas. Resultados concretos evidenciam a transformação de carências em competências e a promoção da resiliência comunitária. \cite{SILVA}

A TCI vai além da promoção da saúde mental, manifestando como um catalisador para a construção de uma sociedade consciente, solidária e participativa. Sua abordagem inclusiva valoriza a diversidade cultural e as capacidades individuais, destacando-se como força motora na formação de sujeitos sociais ativos e conscientes de seu potencial transformador.\cite{BARRETO}

Além disso, a TCI responde à necessidade de uma abordagem psicossocial adaptável a diferentes contextos regionais. Nesses cenários, cada indivíduo reage a situações adaptando-se, seja alterando a própria abordagem, modificando o ambiente ao redor ou combinando essas duas operações em diferentes proporções.\cite{DANTAS} Destaca-se o papel de Adalberto no processo de expansão, desde apresentações em congressos até parcerias que propiciaram sua implantação em diversos países, colaborando com setores governamentais, não governamentais e privados.\cite{GOMES}

Praticada em mais de 24 países nas Américas, Europa e Ásia, a TCI é reconhecida pelo Ministério da Saúde como uma abordagem psicossocial avançada, vai além da esfera clínica, promovendo a construção de redes sociais solidárias, conexões e qualidade de vida. Inserida na Política Nacional de Práticas Integrativas e Complementares(PNPIC) desde 2017, a TCI integra conhecimento científico e sabedoria popular na busca de soluções para conflitos e sofrimentos humanos.\cite{ABRATECOM}

A Associação Brasileira de Terapia Comunitária Integrativa (ABRATECOM), autorizada por seu criador desde 2004, executa um papel fundamental nesse cenário. Atualmente, a ABRATECOM computa mais de 30.500 terapeutas comunitários em todo o Brasil, destacando a amplitude e o alcance dessa prática.\cite{SILVAFRANCO}

\newpage
Os valores que fundamentam a TCI são o acolhimento, a simplicidade, a circularidade do cuidado, a valorização das emoções, a ousadia e transgressão, a geração de dúvidas nas convicções, a horizontalidade das relações, a percepção do outro como recurso, a aceitação da imprevisibilidade e o amor com bom humor.\cite{SILVA}

Em um exemplo específico, durante 20 sessões de Terapia Comunitária Integrativa (TCI) realizadas entre agosto de 2018 e abril de 2019, \cite{BOARETTO} conduziram um estudo quase-experimental, sem distribuição aleatória, para avaliar os níveis de ansiedade e depressão em estudantes de graduação e pós-graduação. Utilizou-se a Escala de “Hospital Anxiety and Depression”, validada em português para o Brasil em 1995. Participaram 25 estudantes, divididos em quatro grupos (28\% pós-graduação, 72\% graduação), durante cinco sessões de TCI, com coleta de dados antes e depois das sessões. Todos eram adultos, com idades entre 19 e 29 anos, sendo a maioria do sexo feminino (84\%). Inicialmente, os escores para possível ansiedade eram de 52\%, e para depressão, de 12\%. Após as sessões de TCI, esses valores reduziram aproximadamente 53.85\% para ansiedade e 66.67\% para depressão.

Ao analisar os dados, comparando os instrumentos aplicados antes e depois das cinco rodas de TCI, é possível verificar que mais da metade dos participantes apresentaram índices menores de ansiedade e depressão. Esse resultado tornou-se evidente ao término do estudo, quando os estudantes relataram uma redução da tensão, medo, inquietude e preocupação, juntamente com um aumento do ânimo, alegria e autoestima. Esses resultados destacam a necessidade e importância de programas de intervenção psicoterápica dentro das universidades.\cite{BOARETTO}

Em outra aplicação prática, indicadora dos benefícios da TCI, foi observada em uma pesquisa realizada para avaliar os efeitos antes e depois de uma sessão. Um caso envolveu uma senhora que enfrentava, de forma recorrente, a difícil situação de ter uma sobrinha envolvida com drogas, sendo sua única família. A sobrinha encorajada a participar da roda de TCI, permaneceu em silêncio durante toda a sessão, mas ao final, expressou que "a terapia não é um lugar de covardes". Sua avaliação antes da sessão refletia um estado emocional fragilizado, enquanto após a TCI, ela se sentia com disposição para explorar novas perspectivas e caminhos. Essa experiência destaca a capacidade transformadora da TCI em situações complexas, proporcionando não apenas um espaço de expressão, mas também potencializando a perspectiva de mudança e crescimento. \cite{LEITEePALOS}




\chapter{Tecnologia na Saúde Mental e na Engenharia Eletrônica}

\section{Discussão sobre o uso da tecnologia na saúde mental.}
\section{Exploração de como a Engenharia Eletrônica pode contribuir para soluções inovadoras em saúde mental.}
