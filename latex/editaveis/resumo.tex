\begin{resumo}
    O presente trabalho de conclusão de curso em Engenharia Eletrônica concentra-se na avaliação do impacto da tecnologia no acesso à Terapia Comunitária Integrativa online. O objetivo principal é desenvolver uma plataforma online utilizando tecnologias como \textit{Next.js}, \textit{TypeScript} e \textit{Tailwind CSS} para facilitar e otimizar o acesso da comunidade às rodas de TCI online. A plataforma proposta terá como função principal o gerenciamento da disponibilidade do terapeuta comunitário e a facilitação do acesso dos participantes. Os terapeutas poderão criar e gerenciar os links de acesso para os grupos comunitários online. Já os usuários e alunos de graduação da Universidade de Brasília terão acesso à plataforma para encontrar e participar de grupos que se adequam à sua rotina, além de uma página de vídeos que incentivam momentos de relaxamento individual.

    O acesso à plataforma será viabilizado por meio da autenticação \textit{OAuth} 2 do Google. A escolha dessa tecnologia permitirá o gerenciamento de calendários por meio de \textit{Application Programming Interfaces} (APIs), bem como a criação de salas de reunião no \textit{Google Meet} destinadas à realização das sessões terapêuticas online. A metodologia adotada neste projeto será pautada em princípios ágeis, buscando utilizar plataformas como serviço (PaaS) para a implementação de uma versão inicial da plataforma de Terapia Comunitária online. Esta versão será então submetida a uma análise estatística detalhada sobre a usabilidade da plataforma, visando a identificação e resolução de problemas, melhorar e manter o desempenho dos sistemas, impulsionando assim a inovação contínua.

    Em sua essência, este trabalho visa harmonizar a tecnologia e a prática dessa abordagem psicossocial avançada, almejando acessibilidade de uma forma intuitiva às rodas online de Terapia Comunitária Integrativa. A proposta busca, portanto, integrar ferramentas tecnológicas para potencializar e democratizar o acesso a abordagens terapêuticas comunitárias.

 \vspace{\onelineskip}
    
 \noindent
 \textbf{Palavras-chave}: TCI. Terapia Comunitária Integrativa. Saúde Mental. Next.js. TypeScript. Tailwind CSS. OAuth2.0. 
\end{resumo}
