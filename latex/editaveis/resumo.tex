\begin{resumo}
    O presente trabalho de conclusão de curso em Engenharia Eletrônica concentra-se na avaliação do impacto da tecnologia na eficácia da Terapia Comunitária Integrativa. O objetivo principal é desenvolver uma plataforma online utilizando tecnologias como Next.js, TypeScript e Tailwind CSS, a fim de facilitar e otimizar as sessões terapêuticas. A plataforma proposta visa servir como um espaço de interação para usuários e terapeutas. Os terapeutas terão a capacidade de criar e gerenciar grupos terapêuticos online, enquanto os usuários e alunos de graduação da Universidade de Brasília terão acesso para encontrar e participar desses grupos. O acesso à plataforma será viabilizado por meio da autenticação OAuth2 do Google. Tal abordagem permitirá o gerenciamento de calendários por meio de APIs, bem como a criação de salas de reunião no Google Meet, destinadas à realização das sessões terapêuticas online. A metodologia adotada neste projeto será pautada em princípios ágeis, buscando utilizar plataformas como serviço (PAAS) para a implementação de uma versão inicial da plataforma. Esta versão será então submetida a uma análise estatística detalhada, visando aprimoramentos contínuos com base nos resultados obtidos. Em sua essência, este trabalho visa harmonizar a tecnologia e a prática psicoterapêutica, almejando oferecer uma abordagem mais acessível e eficaz à Terapia Comunitária Integrativa. A proposta busca, portanto, integrar ferramentas tecnológicas para potencializar e democratizar o acesso a abordagens terapêuticas comunitárias.

 \vspace{\onelineskip}
    
 \noindent
 \textbf{Palavras-chave}: TCI. Terapia Comunitária Integrativa. Saúde Mental. Next.js. TypeScript. Tailwind CSS. OAuth2.0. 
\end{resumo}
